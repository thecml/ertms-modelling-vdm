\documentclass[preprint,12pt]{elsarticle}
\usepackage{amssymb}
\usepackage{lineno}
\usepackage{fixltx2e}
\usepackage{hyperref}
\usepackage{enumitem}
\usepackage{vdmlisting}
\usepackage{float}

\begin{document}

\begin{frontmatter}

\title{Modeling and simulating ERTMS/ETCS using VDM++}

%% use optional labels to link authors explicitly to addresses:
%% \author[label1,label2]{<author name>}
%% \address[label1]{<address>}
%% \address[label2]{<address>}

\author{Christian M. Lillelund (201408354)}

\address{School of Engineering, Aarhus University}

\address{Course: E18 - Modeling of Critical Systems}

\begin{abstract}
Modeling of critical systems is an important tool in software engineering, as it allows us to design a system
from specification and verify its properties in a unambiguous way early in the development process. In this report, we use the object-oriented Vienna Development Method (VDM++) to model, simulate and verify a subset of the specifications for the safety-critical European Rail Traffic Management System (ERTMS) level 2 focusing on those properties of ERTMS that concern interlocking of tracks and safety of the trains. In ERTMS, these features are controlled by the European Train Control System (ETCS) component. With VDM++, we can perform system design analysis and model verification to catch potential faults early before the system is constructed in an implementation language like Java and deployed to a real system. We start by defining requirements and invariants based on the ERTMS/ETCS specification and then use VDM++ to design a formal model followed by model-checking. The abstract implementation of ERTMS detailed in this report provide a simple but fundamental understanding of the signaling system's behavior and how to describe its constraints.
\end{abstract}

\begin{keyword}
ERTMS \sep ETCS \sep VDM++ \sep Formal methods \sep Interlocking \sep Safety
\end{keyword}

\end{frontmatter}

\section{Introduction}
\label{S:introduction}
The railway domain was identified as a grand challenge of computing science in 2004 because it is understandable by the general public, provides useful features in terms of transportation and pose many concerns for design and controllability. One part of the challenge is improving the feasibility and capacity of modern railway as the world's population is increasing and rail traffic now moves cross borders.

The European Rail Traffic Management System (ERTMS) is a signaling and control system developed in the start 2000's to address the interoperability issues with cross-border rail traffic in Europe and lack of capacity with legacy systems. Currently many European countries have their own national stand-alone signaling and control system implemented by a certain set of rules that differ from each country. ERTMS is designed to replace the national systems to make rail transport more frictionless, improve rail capacity and more attractive to consumers. It consists of two primary parts, the European Train Control and Command System (ETCS) to govern the positions and safety of trains, known as the Interlocking (ITL), and the GSM-R radio communications system to send messages between trains and the Radio Block Center (RBC). There a three different levels of ERTMS, L1, L2 and L3. We consider L2 in this report. L2 introduces the RBC, a Eurobalise that register train movement and omits any track side signaling equipment. For a train to enter a new track section, it must requests a movement authority (MA) for that section from the RBC, which will be further explained in section X. ERTMS defines end of authority (EoA) as the point the train is allowed to move to. We consider this aspect as well.

Figure \ref{fig:ertmsoverview} shows the essential components of ERTMS level 2. The train control (EVC) requests movement authorities from the RBC over GSM-R. A typical national rail way will have multiple RBC's for each region. The RBC's communicate with a central interlocking service, that receives the physical location of trains using the Eurobalises, in order to either grant or deny movement authority to a train.

\begin{figure}[h]
	\centering
	\includegraphics[width=0.8\linewidth]{ERTMS.png}
	\caption{An overview of the ERTMS/ETCS level 2 components.}
	\label{fig:ertmsoverview}
\end{figure}

\section{Requirements}
\label{S:requirements}
The purpose of the model is to design and verify the interlocking properties and operating rules of the ETCS system, a part of ERTMS. With our model we aim to show how VDM++ can we used to create and satisfy a subset of the real requirements of ETCS interlocking, abstracting away most of the physical constraints you would have in a real system and focusing on the control mechanisms you find in the RadioBlockCenter and Interlocking components, described in the introduction. Our model should demonstrate a simulated train that interacts with the control system when traversing a set of tracks, as it would in real life, while simultaneously monitoring and adhering to the safety features of the system. We address the following requirements: \\

R1: A train shall not enter a track section which is occupied by another train (same track).\\
R2: A train shall not pass a track boundary without being given a movement authority (MA) to do so.\\
R3: A train shall respect the maximum permitted speed of its current track.\\
R4: Two trains cannot have a MA for the same track at the same time.\\
R5: The system shall not provide an MA for a track that is occupied by a train.\\
R6: The RBC shall not answer MA's for tracks it is not responsible for.\\
R7: The eurobalise shall report train movement to the interlocking system.\\

The requirements and the behavior of the ETCS control architecture, that we elaborate on in section \ref{S:systemarch}, are inspired from literature X and X, that describe, model and verify many of the real functional properties of ERTMS. In this report we choose a portion of these to model and test our requirements \textit{R1-R8}. UML will be used to show some of the logical VDM++ classes and their relationships. From the specification of ERTMS we can define a typical usage scenario for a train that wishes to enter a track. Given a train \textit{T1}, a track TR1 from a set of tracks (i=1,..,n), an \textit{RBC} responsible for tracks \textit{TR\textsubscript{n}}, a eurobalise \textit{Eb} mounted on the track that registers physical movement and a central interlocking service \textit{ITL}.

\begin{itemize}[noitemsep]
	\item \textit{T1} wants to enter track \textit{TR1}. \textit{T1} requests MA for \textit{TR1} from RBC.
	\item RBC contacts the ITL to verify the request.
	\item RBC grants MA to train \textit{T1} for track TR1.
	\item \textit{T1} enters track TR1.
	\item Eb registers movement and informs ITL.
\end{itemize}

This functionality will be further explored in the model architecture and design.

\section{Model architecture}
\label{S:systemarch}

The architecture of the model represents the entire system as entities, that are later defined in VDM++ as classes, that encapsulate functionality related to the individual entity. We consider the purpose of our model, that is to express and simulate several important aspects of ERMTS/ETCS, and use the primary components of such system as our entities. These can seen in figure \ref{fig:ertmscontrolarchitecture} that show a way to model the control architecture of ERTMS based in the real specification set for ERTMS level 2 detailed in X, but junction control and route guidance has been omitted.

\begin{figure}[h]
	\centering
	\includegraphics[width=0.8\linewidth]{ERTMSControl.png}
	\caption{The ETCS control architecture partly from the real specification.}
	\label{fig:ertmscontrolarchitecture}
\end{figure}

We shall describe the components in more detail before designing our model. The \textbf{controller} places route requests into the interlocking at certain time \textit{t} intervals. The point is to control the flow of traffic and how many routes can be available concurrently in the rail network, a way to limit congestion. A route is simply a set of tracks. \textbf{Interlocking} receives route requests and stores them in a control table containing available routes. It receives track occupation events from physical units along the track side and uses this information to update the control table with available routes. The \textbf{RadioBlockCenter} receives MA requests from trains, checks route availability and forwards these to the interlocking. RBC's are responsible for multiple tracks, i.e. a block. The \textbf{Train} sends MA requests to the RBC before entering a new track in a route. For each track, a line segment with a geographic starting and terminal point, the train must request and obtain a MA. This is also stated in the requirements. Each track has a certain speed limit, and the train must never exceed this. The \textbf{Eurobalise} registers physical movement when trains enter and leave tracks and conveys this to the interlocking. 

In a real scenario, the eurobalise sends track and speed information to passing trains and the ETCS would monitor this and eventually slow down trains, if they go too fast. We do not consider this part in our model. The ETCS architecture in figure \ref{fig:ertmscontrolarchitecture}, requirements and functionality will be used to create VDM++ classes, functions and invariants in the next section.

\section{Model design}

In this section we will review the model in VDM++ in a top-down approach and elaborate on interesting design choices made. The entities in our architecture are now VDM classes, types represent information state objects, and events and requests are described with operations and functions. A full class diagram can be found in Appendix A.

\subsection{Controller}

The controller controls at what rate new routes are placed into the interlocking. In a real setting, it regulates the flow of trains and adds a detail of security, as new routes that become available (free of trains) are not immediately placed into the interlocking. Listing \ref{lst:controller} shows the how we do this in our VDM model. Operation SendRouteRequest() is executed sequentially, picks a random route from a sequence of routes and requests the route at the interlocking.

\begin{vdmsl}[label=lst:controller,caption=Definitions and SendRouteReques() operation in the Controller.]
	class Controller
	
	instance variables
	private itl : Interlocking;
	private routes : seq of Interlocking`Route;
	
	operations
	public Step:() ==> ()
	Step() == SendRouteRequest();
	
	private SendRouteRequest:() ==> ()
	SendRouteRequest() ==
	(dcl rn : nat := MATH`rand(len routes)+1;
		itl.RequestRoute(routes(rn));
	);
	...
	end Controller
\end{vdmsl}

\subsection{Interlocking}

The interlocking has three primary operations with core functionality, defined as such because they change the state of the class. First operation accepts new route requests from the controller for routes $R_{n}$ and place it into its locking table (available routes) if it not already occupied. We choose a set for these, as they are finite, the order is not important and duplicate elements should not be significant. Sets also support some useful operators which we will see. The second operation receive and store information from corresponding Eurobalises (track trackers) about the physical movement of trains. We choose a mapping for these, since they define lookup operators that allow us to assert if a particular track is occupied or not. The $trackTable$ maps tracks with a given coordinate to either a true or false value, if the track is occupied or not. Third operation respond to the RadionBlockCenter requesting a MA (Movement Authority) on the behalf of trains. We start be defining the implemented types and instance variables (listing \ref{lst:itl1}).

Notice the atomic use in the constructor to defer invariant assertion after the assignments.

\begin{vdmsl}[label=lst:itl1,caption=Types and state information and constructor for the Interlocking class.]
	class Interlocking
	...
	types
	public Order = <PROCEED_GRANTED> | <PROCEED_DENIED>;
	public ProceedReply :: message : Order
	routesAvaliable : set of Route;						 
	public Track :: startX : nat
	startY : nat
	endX : nat
	endY : nat
	description : seq of char
	maxSpeed : nat;
	public Route = set of Track;
	
	instance variables
	private trackTable : map Track to bool;
	private availableRoutes : set of Route;
	inv InvNoDuplicateTrack(availableRoutes);
	inv InvNoTrackAvailableAndOccupied(trackTable,
	 availableRoutes);
	
	public Interlocking: set of Route ==> Interlocking
	Interlocking(rts) ==
	atomic (
		availableRoutes := rts;
		trackTable := { tr |-> false |
			 tr in set GetTracksInRoutes(rts)};
	);
	...
	end Interlocking
\end{vdmsl}

The interlocking has two invariants, $InvNoTrackAvailableAndOccupied$ that makes sure that occupied tracks in $trackTable$ and set of available routes are mutually exclusive and $InvNoDuplicateTrack$ that makes sure no duplicate track (with the same start and end coordinates) exists in the set of available routes (listing \ref{lst:itl2})

\begin{vdmsl}[label=lst:itl2,caption=Two invariants that verify the integrity of the available routes and checks a route for duplicate track.]
	public InvNoTrackAvailableAndOccupied: map Track to bool
	 * set of Route -> bool
	InvNoTrackAvailableAndOccupied(trmap, rts) ==
	forall rt in set rts & forall tr in set rt
		& tr not in set dom trmap or trmap(tr) = false;
	
	public InvNoDuplicateTrack: set of Route -> bool
	InvNoDuplicateTrack(srt) ==
	forall rt1 in set srt &
		forall tr1,tr2 in set rt1 & tr1 <> tr2
		=> tr1.startX <> tr2.startX or tr1.endX <> tr2.endX 
		or tr1.startY <> tr2.startY or tr1.endY <> tr2.endY;	
\end{vdmsl}

Next is the $RequestRoute()$ operation called by the controller (listing \ref{lst:itl3}). If the route $rt$ is not in available routes and tracks are not occupied, we add it to $availableRoutes$ and clear its track. $GetTracksInRoutes()$ is a utility operation that returns the tracks in a route.

\begin{vdmsl}[label=lst:itl3,caption=Definition of the SendRouteReques() operation.]
	public RequestRoute: (Route) ==> ()
	RequestRoute(rt) ==
	(
	if (rt not in set availableRoutes and
		forall tr in set rt & tr not in set dom trackTable
		 or trackTable(tr) = false)
		 then (
			availableRoutes := availableRoutes union {rt};
			trackTable := trackTable ++ { tr |-> false |
		 				  tr in set GetTracksInRoutes({rt})};	
		 ) else skip;
	)	
\end{vdmsl}

The next operation $SetTrackState()$ is called by the Eurobalise when trains enter a new track $tr$ or leaves a track. This is passed as an argument by the Eurobalise. We set the track as occupied on enter and clears it on leave in the $trackTable$ (listing \ref{lst:itl4}). We set a precondition that the track must be known to the interlocking.

\begin{vdmsl}[label=lst:itl4,caption=Definition of the SendRouteReques() operation.]
	public SetTrackState: Track * Eurobalise`TrainState ==> ()
	SetTrackState(tr, sta) ==
	if(sta = <TRAIN_ENTER>) then
		trackTable := trackTable ++ {tr |-> true}
	else if (sta = <TRAIN_LEAVE>) then
		trackTable := trackTable ++ {tr |-> false}
	pre tr in set dom trackTable;	
\end{vdmsl}

Finally the $RequestToProceed()$ operation called by the RadionBlockCenter (listing \ref{lst:itl5}). It returns a proceed granted if the route $ptr$ is in $availableRoutes$ and for all tracks in route $ptr$ neither are in the $trackTable$ as occupied. Before a proceed is granted, it removes the route $prt$ from the list of available routes as well all other routes that may contain the same track as $ptr$.

\begin{vdmsl}[label=lst:itl5,caption=Definition of the SendRouteReques() operation.]
	public RequestToProceed: Route * set of Track ==> ProceedReply
	RequestToProceed(prt, respTrs) ==
	if (card availableRoutes > 0) then (
		if(exists rt in set availableRoutes
		& prt subset rt and forall tr in set prt
		& tr in set dom trackTable) then
			if(forall tr in set prt & trackTable(tr) = false) then (
				availableRoutes := {rts | rts in set availableRoutes
				& forall ptrs in set {prt} & rts inter ptrs = {}};
				return ProceedGranted(respTrs);
			)
			else return ProceedDenied(respTrs)
		else return ProceedDenied(respTrs)
	) else ( return ProceedDenied(respTrs) )
	pre card prt > 0;
\end{vdmsl}

\subsection{RadioBlockCenter}

The RadioBlockCenter receives MA requests from trains and forwards these to the interlocking, basically acting as a gateway but with its own state of available routes and responsible tracks initialized in the constructor. Similar to the interlocking, we use sets here as the order is unimportant and duplicate should not be significant for our model. Also at no point do we need to select certain elements, but operators like membership, intersection and union are important to us. Listing \ref{lst:rbc1} shows the declarations for the RadioBlockCenter.

\begin{vdmsl}[label=lst:rbc1,caption=Definition of class state and constructor for RadioBlockCenter.]
	class RadioBlockCenter
	types
	public MovementAuthorityReply =
	 <MovementAuthorityGranted> | <MovementAuthorityDenied>;
	
	instance variables
	private respTracks : set of Interlocking`Track;
	private availableRoutes : set of Interlocking`Route := {};
	private itl : Interlocking;
	
	public RadioBlockCenter : set of Interlocking`Track
	* Interlocking ==> RadioBlockCenter
	RadioBlockCenter(trs,pitl) ==
		respTracks := trs;
		itl := pitl;
	);
	...
	end RadioBlockCenter
\end{vdmsl}

To receive the MA's, we define a operation $RequestMovementAuthority()$ that takes a certain route $Rt$, checks that the radio block center is responsible for tracks in $Rt$, then forwards a synchronous proceed request to the interlocking and sends the reply back to the train for route $Rt$. The interlocking replies with all available routes that intersects with the responsible tracks, hence it is passed as a parameter (listing \ref{lst:rbc2})

\begin{vdmsl}[label=lst:rbc2,caption=Definition of the public.]
	RequestMovementAuthority: Interlocking`Route
	 ==> MovementAuthorityReply
	RequestMovementAuthority(rt) ==
	(
	if(rt subset respTracks) then (
		dcl msg : Interlocking`Order;
		def mk_Interlocking`ProceedReply(message,rtr)
			 = itl.RequestToProceed(rt, respTracks)
		in ( msg := message; availableRoutes := rtr; );
			if (msg = <PROCEED_GRANTED>)
				then ( return <MovementAuthorityGranted>; )
			else (
				return <MovementAuthorityDenied>;
			)
		) else return <MovementAuthorityDenied>;
	);
\end{vdmsl}

\subsection{Eurobalise}

The Eurobalise is a piece of track side equipment that registers the physical movement of trains, as described previously. The class is straightforward, as it simply defines some track it is situated at and the global interlocking system (listing \ref{lst:eb1}).

\begin{vdmsl}[label=lst:eb1,caption={Types, variables, constructor for the Eurobalise.}]
class Eurobalise
types
public TrainState = <TRAIN_ENTER> | <TRAIN_LEAVE>;

instance variables
private itl : Interlocking;
private track: Interlocking`Track;

public Eurobalise : Interlocking * Interlocking`Track
 ==> Eurobalise
Eurobalise(pitl, tr) ==
(
itl := pitl;
track := tr;
);
end Eurobalise			
\end{vdmsl}

The two operations $Enter()$ and $Leave()$ invokes the interlocking for a track $track$ when a train either enters or leaves it (listing \ref{lst:eb2}.

\begin{vdmsl}[label=lst:eb2,caption={The two operations that trains call to register their movement.}]
	public Enter: () ==> nat1
	Enter() == (
		itl.SetTrackState(track, <TRAIN_ENTER>);
		return track.maxSpeed;
	);
	
	public Leave: () ==> ()
	Leave() == (
		itl.SetTrackState(track, <TRAIN_LEAVE>);
	);	
\end{vdmsl}

\subsection{Train}

The Train is the driver of the system and each created has a route table with one to many routes that it must traverse. When starting, it invokes the RadioBlockCenter by sending a MA request for the first route $Rt$ in its route table, and if the MA is granted, it will enter the first track in the route by calling the Eurobalise, set its current speed to match the track, move onto the end and leave it by calling the Eurobalise. The train will halt should it not obtain a MA for any $Rt$. In our simulation we follow the movement of trains by printing a log of their X and Y positional coordinates. The train has a lot of state declarations, so for clarity, we show just the sequential $Drive()$ function in listing \ref{lst:train1}. The full code for Train is found in Appendix A.

\begin{vdmsl}[label=lst:train1,caption={Drive() moves the train along a sequence of routes, requests MA for each route and obeys track speed limits.}]
	private Drive: () ==> ()
	Drive() ==
	(
	if (state = <Running> or state = <WaitingForSignal>
		and len routeTable > 0) then (
			dcl currentRoute : Interlocking`Route := hd routeTable;
			
			if(rbc.RequestMA(currentRoute) = <MAGranted>)
			then (
				state := <Running>;
				for track in GetTracksInRoute(currentRoute) do (
					dcl currentEb : Eurobalise := transponders(track);
					currentSpeed := currentEb.Enter();
					posX := track.endX;
					posY := track.endY;
					currentEb.Leave();
				);
			routeTable := tl routeTable;
			if (routeTable = []) then (
				currentSpeed := 0;
				state := <Finished>;
			);
		) else (
		currentSpeed := 0;
		state := <WaitingForSignal>;
		)
	) else skip;
	);
\end{vdmsl}

\section{Unit and combinatorial testing}

A typical unit test verifies the functional properties of a class's functions and operations, often conducted in isolation with stubs for any external dependencies. A unit test invokes the unit under test (a class) with some input and evaluates whether the output satisfies what was expected. VDM++ offers unit testing options with VDMUnit, which we have used to test the four control classes in our system and shall explore in a bit. Additionally VDM++ provides a framework for combinatorial testing used to verify pre, post and invariant conditions in our classes. Both methodologies have been applied in this project and the complete set of sets can be found in the attached code, see Appendix A. We start by creating the initial environment for the tests seen in listing \ref{lst:itlunit1}.

\begin{vdmsl}[label=lst:itlunit1,caption={The test uses some predefined values and initiates the UUT in the constructor.}]
	class InterlockingTest is subclass of TestCase
	values
	routes = {{mk_Interlocking`Track(10,5,15,5,"AB",100)},
		{mk_Interlocking`Track(15,5,20,5,"BC",110)},
		{mk_Interlocking`Track(20,5,25,5,"CD",110)},
		{mk_Interlocking`Track(25,5,30,5,"DE",110)},
		{mk_Interlocking`Track(25,5,30,5,"EF",110)}};
	
	instance variables
	private uut: Interlocking;

	operations
	public InterlockingTest: () ==> InterlockingTest
	InterlockingTest() ==
		uut := new Interlocking(routes);
\end{vdmsl}

We proceed to define some functional unit tests made for the Interlocking class in listing \ref{lst:itlunit2}, where two examples are shown.

\begin{vdmsl}[label=lst:itlunit2,caption={Two unit tests defined for the Interlocking class.}]
	public Test_RouteInAvailableRoutes_ProceedGranted: () ==> ()
	Test_RouteInAvailableRoutes_ProceedGranted() ==
	(
	dcl testRoute : Interlocking`Route
	 := {mk_Interlocking`Track(10,5,15,5,"AB",100)};
	dcl msg : Interlocking`ProceedReply
	 := uut.RequestToProceed(testRoute, testRoute);
	assertTrue(msg.message = <PROCEED_GRANTED>);
	);
	
	public Test_TrainEntersAndLeavesTrack_TrackChangesState: ()
	 ==> ()
	Test_TrainEntersAndLeavesTrack_TrackChangesState() ==
	(
	dcl track : Interlocking`Track
	 := mk_Interlocking`Track(10,5,15,5,"AB",100);
	uut.SetTrackState(track, <TRAIN_ENTER>);
	assertTrue(track in set uut.GetTrackTable());
	uut.SetTrackState(track, <TRAIN_LEAVE>);
	assertTrue(track not in set uut.GetTrackTable());
	);
\end{vdmsl}

The combinatorial tests are defined in the classes and can be seen to the full extend in Appendix A. With these tests we verify preconditions, postconditions and state invariants. As opposed to the unit tests, that fail if any of these are violated, the combinatorial tests print inconclusive in such cases and succeeds no matter the assertion was true or false. Again looking at the interlocking, listing \ref{lst:itlcomb1} defines four combinatorial tests that check our invariants. The results are omitted here, but can be viewed in Appendix B. All tests can be seen in Appendix A.

\begin{vdmsl}[label=lst:itlcomb1,caption={Four combinatorial tests that exercise our invariant functions in the Interlocking class.}]
	T1: let trmap in set
	 {{mk_Track(10,5,15,5,"AB",100) |-> false}} in
	InvNoTrackAvailableAndOccupied(trmap, routes);
	
	T2: let trmap in set
	 {{mk_Track(10,5,15,5,"AB",100) |-> true}} in
	InvNoTrackAvailableAndOccupied(trmap, routes);
	
	T3: let rts in set {{{mk_Track(5,5,10,5,"AB",100),
		mk_Track(2,5,10,5,"BA",100)},
		{mk_Track(20,5,10,5,"AB",100),
		mk_Track(30,5,10,5,"DE",100)}}} in
			InvNoDuplicateTrack(rts);
	
	T4: let rt in set {{mk_Track(10,5,15,5,"AB",100),
			mk_Track(15,5,20,5,"BC",100)},
			{mk_Track(20,5,25,5,"AB",100),
			 mk_Track(25,5,30,5,"DE",100)}} in
	let trmap = {mk_Track(10,5,15,5,"AB",100) |-> false} in
	InvIsTrackOccupied(rt, trmap);
\end{vdmsl}

\section{Simulation}

\section{Discussion}

\section{Conclusion}

%% The Appendices part is started with the command \appendix;
%% appendix sections are then done as normal sections
%% \appendix

%% \section{}
%% \label{}

%% References
%%
%% Following citation commands can be used in the body text:
%% Usage of \cite is as follows:
%%   \cite{key}          ==>>  [#]
%%   \cite[chap. 2]{key} ==>>  [#, chap. 2]
%%   \citet{key}         ==>>  Author [#]

%% References with bibTeX database:

\bibliographystyle{model1-num-names}
\bibliography{sample.bib}

%% Authors are advised to submit their bibtex database files. They are
%% requested to list a bibtex style file in the manuscript if they do
%% not want to use model1-num-names.bst.

%% References without bibTeX database:

% \begin{thebibliography}{00}

%% \bibitem must have the following form:
%%   \bibitem{key}...
%%

% \bibitem{}

% \end{thebibliography}


\end{document}

%%
%% End of file `elsarticle-template-1-num.tex'.
